\documentclass[
fontsize=12pt, % Schriftgrösse
twoside, % zweiseitig (opt: oneside)
headings=small, chapterprefix=false, numbers=noenddot, toc=indent, chapterentrydots=true] {scrbook} % vgl. Optionen in der Koma script Dokumentation

\date{05.08.2020}
\author{Marc Bayard}
\title{Header Vorlage}
% Vorlage für einen Header für den LaTeX-Workshop vom 05.08.2020 an der ZHB Luzern


% Allgemeine Pakete und Befehle
%%%%%%%%%%%%%%%%%%%%%%%%%%%%%%%%%%%%%%%%%%%%%%%%%%%%%

%Sprachen und Zeichen
\usepackage[T1]{fontenc}
\usepackage[utf8]{inputenc}
\usepackage[ngerman]{babel}

\usepackage{verbatim} % für mehrzeiligen %-Kommentar: \begin{comment} ... \end{comment}
\usepackage[style=german]{csquotes} % Anführungszeichen global wechseln, im Text: \enquote{Text}, \enquote*{Text} erzeugt einfache Anführungszeichen
\usepackage[table]{xcolor} % Farben laden
\usepackage{rotating} % für vertikalen Text: \begin{rotate}{90} Text \end{rotate}
\usepackage{hyperref} % Für Querverweise aller Art, verweist auf \label{}
% \pageref für die Seitenzahl, \ref für den Zähler der nächsthöheren Überschrift \nameref für den Titel der nächsthöheren Überschrift (Kurztitel, falls vorhanden)
% Beispiel für einen Seitenverweis: Bei "siehe auf Seite \pageref{These}" wird die Seitenzahl der Seite eingefügt auf der \label{These} steht.
% Beispiel für selbstdefinierten Verweisbefehl: \newcommand{\myref}[1]{\footnote{Siehe Kapitel \ref{#1}: \enquote{\nameref{#1}} auf S.~\pageref{#1}.}} Im Text wird mit \myref{These} eine Fussnote mit dem angegebenen Text und den Verweisen auf \label{These} erstellt.

% Seitenlayout
\usepackage[
a4paper,
twoside, % zweiseitig
inner=2.5cm, % Rand innen
outer=3cm, % Rand aussen
top=2cm, % Rand oben
bottom=2cm, % Rand unten
includeheadfoot, % Kopf- und Fusszeilen innerhalb der Ränder
headheight=1cm, % Höhe der Kopfzeile
headsep=0.5cm, % Abstand von Kopfzeile zu Text
footskip=1.5cm, % Abstand von Text (inkl. Fussnoten) zu Fusszeile
]{geometry}

%\setlength\parindent{0pt} % Damit werden die ersten Zeilen eines Absatzes nicht eingerückt
\RequirePackage[ngerman=ngerman-x-latest]{hyphsubst} % für bessere Trennung (wird laufend verändert...)
\sloppy
\raggedbottom % Damit die Absätze zusammen bleiben und nicht gedehnt werden
\clubpenalty = 10000
\widowpenalty = 10000
\displaywidowpenalty = 10000 % Diese drei sind gegen Hurensöhne und Schusterjungen
%\hyphenpenalty=50  % Wert für das Unterdrücken der Silbentrennung
%\tolerance=100     % Wert für die Toleranz, Abstände zwischen den Wörtern zu variieren
\frenchspacing % Damit die Abstände nach dem Satzende nicht grösser als normal sind, in KOMA Script eigentlich Standard

% Befehle der Textauszeichnung
\newcommand{\zit}{\enquote} % normales Zitat in doppelten Anführungszeichen
\newcommand{\zite}{\enquote*} % Zitat im Zitat in einfachen Anführungszeichen
% längeres Zitat siehe unten "Umgebungen"
\newcommand{\wt}{\enquote*} % Werktitel in einfachen Anführungszeichen
\newcommand{\uwg}{\enquote*} % uneigentlicher Wortgebrauch in einfachen Anführungszeichen, Bsp.: "Goethe liebte das ,Sowohl-als-auch' der Sprache."
\newcommand{\lat}{\textit} % lateinische Begriffe kursiv
\newcommand{\bet}{\textit} % Betonung kursiv, Bsp.: "Er hat die Frage zwar nicht /gelöst/, aber er hat sie /erlöst/, gelöst werden zu müssen."

% Schriftart
\usepackage{libertine}
%\usepackage{emgaramond}



% Gliederung
%%%%%%%%%%%%%%%%%%%%%%%%%%%%%%%%%%%%%%%%%%%%%%%%%%%%%

% Eingabe direkt im Text: \Gliederungsbefehl[Kurztitel]{Überschrift}
% - \Giederungsbefehl: definiert die Ebene, z.B. \chapter; bei \chapter* wird kein Zähler erzeugt
% - [Kurztitel]: Für Inhaltsverzeichnis und die automatischen Kopfzeilen (automark) kann man die Gliederungsüberschriften kürzen (optional)
% - {Überschrift}: Selbstgewählter Titel (ohne Zähler, der wird automatisch hinzugefügt)

% Tiefe der Nummerierung in der Gliederung, (part=-1)
\setcounter{secnumdepth}{3} % 3 bedeutet also Gliederung bis Subsubsection

% Die Zähler
\renewcommand{\thepart}{\arabic{part}}
\renewcommand{\thechapter}{\Roman{chapter}}
\renewcommand{\thesection}{{\thechapter}.~\arabic{section}}
\renewcommand\thesubsection{\thesection .~\arabic{subsection}}
\renewcommand\thesubsubsection{}
\renewcommand\theparagraph{}

% Den Zahlen eines Zählers Wörter zuordnen. Achtung!: Der Zähler (hier: \thepart) muss \arabic sein!
\usepackage{datetime}
\def\ordinal{\ifcase \thepart\or Erster\or Zweiter\or Dritter\or Vierter\or Fünfter\or Sechster\fi}

% Formatierung der Gliederungsüberschriften
\renewcommand{\partformat}{\thispagestyle{empty}\mdseries\rmfamily\upshape\Huge\ordinal~Teil} % \ordinal ist oben definiert
\setkomafont{part}{\thispagestyle{empty}\mdseries\rmfamily\centering\upshape\Huge}
\renewcommand{\chapterformat}{\mdseries\rmfamily\upshape\Large\thechapter ~\,}
\setkomafont{chapter}{\Large\selectfont\mdseries\rmfamily\centering\upshape}
\setkomafont{section}{\mdseries\rmfamily\upshape\large}
\setkomafont{subsection}{\mdseries\rmfamily\upshape\normalsize}

%\renewcommand*{\raggedsection}{\centering} % Überschriften global zentrieren
\setkomafont{disposition}{\selectfont} % Damit in Überschriften dieselbe Schriftart ist.

% Kosmetik bei den vertikalen Abständen
\RedeclareSectionCommand[
  beforeskip=-2\baselineskip,
  afterskip=1\baselineskip]{section}
\RedeclareSectionCommand[
  beforeskip=-1\baselineskip,
  afterskip=.5\baselineskip]{subsection}

% damit die Chapterzählung in jedem part neu beginnt
 %\makeatletter
 %\@addtoreset{chapter}{part}
 %\makeatother

% für durchgehende Nummerierung (z.B. bei §)
 %\usepackage{chngcntr}
 %\counterwithin{chapter}{part}



%Kopf- und Fusszeile
%%%%%%%%%%%%%%%%%%%%%%%%%%%%%%%%%%%%%%%%%%%%%%%%%%%%%

\usepackage{scrlayer-scrpage}
\pagestyle{scrheadings}
\clearpairofpagestyles % alles Bisherige löschen

% Inhalte
\automark[section]{chapter} % automatisch Überschriften in der Kopfzeile wählen (rechte Seite und linke Seite)
\chead{\headmark} % Einbinden der automatischen Überschriften in Mitte Kopfzeile
\ofoot{\pagemark} % Seitenzahlen in Aussen Fusszeile
\makeatletter
\ifoot{Autor: \@author \\Datum: \@date} % Einbinden von Autor und Datum in Innen Fusszeile
\makeatother
% Die Möglichkeiten der Positionierungen ergeben sich also aus der Kombination von i/c/o und head/foot.

% Zähler weglassen:
%\renewcommand*{\chaptermarkformat}{}
%\renewcommand*{\sectionmarkformat}{}

% Formatierung
\setkomafont{pagehead}{\footnotesize\textup} % Formatierung der Kopfzeile
\setkomafont{pagenumber} {\small % Formatierung der Seitenzahl
%\rightline % bei oneside
}

% Eigener Stil für Seiten mit Kapitelüberschrift
\renewcommand\chapterpagestyle{empty} % empty = leere Kopf- und Fusszeile



%Fussnoten
%%%%%%%%%%%%%%%%%%%%%%%%%%%%%%%%%%%%%%%%%%%%%%%%%%%%%

% Kurzbefehl
\DeclareRobustCommand{\f}{\footnote}

% Linie vor Fussnoten
\setfootnoterule{5cm}
\newlength\flinie
\setlength\flinie{12.5cm} % Ergibt eine Linie von 5 cm, keine Ahnung, warum der Faktor 2.5x
\makeatletter
\setlength{\skip\footins}{22.4pt}% woher die .4 kommen sieht man gleich
\renewcommand\footnoterule{%
\kern-12.4\p@% Platz unter der Linie + Liniendicke
\hrule\@width.4\flinie% Länge ggf. anpassen
\kern12\p@}% Platz unter der Linie
\makeatother

% Formatierung
\usepackage[hang, % Layout wie Nummerierung
marginal, % Einzug, mit \setlength{\footnotemargin}{...}
multiple, % Mehrere Fussnoten im Text nacheinander möglich
% norule, %keine Linie
stable, % Fussnoten in den Gliederungsüberschriften möglich,
bottom % raggedbottom für Fussnoten
% perpage % für Neunummerierung auf jeder Seite
]
{footmisc}
\setlength{\footnotemargin}{0.6cm} %Einzug
\setlength{\footnotesep}{7.5pt} % Abstände zwischen den Fussnoten?
\interfootnotelinepenalty=10000 % Verhindert das Fortsetzen von Fussnoten auf der gegenüberliegenden Seite

% Formatierung der Fussnotenzähler:
\deffootnote{0.6cm}{0.3cm}{\makebox[0.6cm][l]{\thefootnotemark}}

%\usepackage{remreset} % damit die Nummerierung nach section neu anfängt, und nicht nach chapter
%\makeatletter
%\@removefromreset{footnote}{chapter}
%\@addtoreset{footnote}{section}
%\makeatother


%Zeilenabstand
%%%%%%%%%%%%%%%%%%%%%%%%%%%%%%%%%%%%%%%%%%%%%%%%%%%%%

\usepackage{setspace} % ACHTUNG!: muss nach \usepackage{footmisc} eingebunden werden.
% Direkt im Text: “\singlespacing”, “\onehalfspacing” oder “\doublespacing”

\begin{comment}
% Gleich 1.5-Zeilenabstand in Word (sinnlos, aber vielleicht eines Tages nützlich)
\makeatletter
\newcommand{\MSonehalfspacing}{%
  \setstretch{1.44}%  default
  \ifcase \@ptsize \relax % 10pt
    \setstretch {1.448}%
  \or % 11pt
    \setstretch {1.399}%
  \or % 12pt
    \setstretch {1.433}%
  \fi
}
\newcommand{\MSdoublespacing}{%
  \setstretch {1.92}%  default
  \ifcase \@ptsize \relax % 10pt
    \setstretch {1.936}%
  \or % 11pt
    \setstretch {1.866}%
  \or % 12pt
    \setstretch {1.902}%
  \fi
}
\makeatother
\end{comment}



%Umgebungen
%%%%%%%%%%%%%%%%%%%%%%%%%%%%%%%%%%%%%%%%%%%%%%%%%%%%%

% Zitat
\renewenvironment{quote} % Links und rechts Einzüge anpassen
   {\list{}{\leftmargin.4cm\rightmargin0cm\parsep0pt\topsep.2cm plus.2cm minus.1cm}\item\relax}
   {\endlist}

\makeatletter % Damit noindent (kein Einzug) danach, auch nicht mit leerer Zeile dazwischen
\newcommand*{\@doendeq}{%
  \everypar{{\setbox\z@\lastbox}\everypar{}}}

\newenvironment{quoter}{\begin{quote} % Neue quote-Umgebung mit eigener Formatierung definieren
\singlespacing\small\selectfont}{\end{quote}\ignorespacesafterend\par\noindent\aftergroup\@doendeq}
\makeatother
% Aufrufen im Text mit \begin{quoter} Text... \end{quoter}


% Bilder
\usepackage{graphicx}
\graphicspath{ {././Bilder/}}
% Bild im Text einbinden mit
% \begin{figure}[h]\centering % h ist für hier, statt z.B. unten am Rand
% \includegraphics[width=12cm]{Bilder/Bilddatei.jpg}
% \caption{Bildbeschriftung}
% \end{figure}

% Bildbeschriftung
\usepackage[
%labelformat=empty, % ohne "`Abb:"'
textformat=period, % Punkt am Schluss
justification=centerlast, % letzte Zeile zentriert
font={scriptsize, it, singlespacing}, % Formatierung
width=12cm, % Breite
skip=0.2cm % Abstand zum Bild
]{caption}

\usepackage{chngcntr}
\counterwithout{figure}{chapter}
\counterwithout{table}{chapter}


\renewcaptionname{ngerman}{\figurename}{Abb.} % damit kann man das label Abb: ändern
\renewcaptionname{ngerman}{\listfigurename}{Abbildungen} % Titel des Abbildungsverzeichnisses, wird mit \listoffigures im Text eingebunden


% Tabelle
\usepackage{multirow} % für mehrzeilige Zellen
\usepackage{array}
\renewcommand{\arraystretch}{1.5} % für Abstand zwischen Zeilen und Linien

% Tabelle normal, direkt im Text:
% \begin{tabular}{|p{1cm}|p{11cm}|p{2.5cm}|}
% \firsthline
% a & b & c \\ [Durchschuss, wenn nötig, z.B.: 0.2cm]
% \hline
% d & e & f \\ [0.2cm]
% \lasthline
% \end{tabular}

% Tabelle für erweiterte Bedürfnisse:
\usepackage{tabularx}
% direkt im Text:
% \begin{tabularx}{\textwidth}{@{}|X|X|X|X|@{}} % statt X geht auch c oder r oder dann p{0.5\textwidth} (siehe Dokumentation) oder Y (bzw. selbstdefiniert, siehe unten), @{} verhindert Abstände
% a & b & c & d \\ [Durchschuss]
% \hline
% \end{tabularx}

% \newcolumntype{Y}{>{\centering\arraybackslash}X} % selbstefinierter Spaltentyp mit eigener Formatierung (hier: zentriert)



%Inhaltsverzeichnis
%%%%%%%%%%%%%%%%%%%%%%%%%%%%%%%%%%%%%%%%%%%%%%%%%%%%%

% “\tableofcontent” für das Erzeugen des Inhaltsverzeichnisses im Hauptdokument
\setcounter{tocdepth}{3} % Tiefe der angezeigten Gliederung im Inhaltsverzeichnis
\addtokomafont{sectioning}{\rmfamily} % Damit auch im Inhaltsverzeichnis dieselbe Schrift ist

% Part Überschriften (Gebastel für erweiterte Bedürfnisse ;-)
\renewcommand*{\addparttocentry}[2]{\addtocentrydefault{part}{\protect\parbox{\textwidth\scshape\large\mdseries}{\hspace*{\fill}\ordinal~\partname\hspace*{\fill}\\ \hspace*{\fill}#2\hspace*{\fill}}}{}} % zentrieren und mit 2 Zeilen: \ordinal und \partname (hier: “n-ter Teil”) sind oben bei Gliederung definiert, sie ersetzten das Argument #1; #2 ist die jeweilige Überschrift; \centering verursacht Probleme, darum die Lösung mit \hspace*{\fill}, das auf beiden Seiten des Textes gleichmässig mit leerem Platz auffüllt.
\setkomafont{partentrypagenumber}{\color{white}} % Keine Seitenzahl bei part (zugegeben, ein unsauberer Kniff. Wer kennt eine bessere Lösung?)

% Einzüge der Überschriften (von Rand bis Nummerierung, von Nummerierung bis Text)
\RedeclareSectionCommand[tocindent=0cm,tocnumwidth=0.7cm]{chapter}
\RedeclareSectionCommand[tocindent=0.7cm,tocnumwidth=1.2cm]{section}
\RedeclareSectionCommand[tocindent=1.9cm,tocnumwidth=1.4cm]{subsection}
\RedeclareSectionCommand[tocindent=0cm,tocnumwidth=0.5cm]{subsubsection}



%Literaturverzeichnis
%%%%%%%%%%%%%%%%%%%%%%%%%%%%%%%%%%%%%%%%%%%%%%%%%%%%%

%Literaturverzeichnis


\usepackage[style=verbose-ibid,backend=biber,bibencoding=utf8,bibwarn=true,language=auto,autolang=none,block=none,backrefstyle=two,
%sorting=Reihenfolge,
citepages=omit,ibidpage=true,isbn=false,url=false,doi=false]{biblatex}

\addbibresource{Literatur/Literatur.bib}



%\DeclareBibliographyCategory{nicht-in-bib}    % Für Einträge im Text, aber nicht im LV
%\addtocategory{nicht-in-bib}{EB,Dud,CD}
% dann beim LV: \printbibliography[notcategory=nicht-in-bib]

%\usepackage{ragged2e}
%\renewcommand*{\bibsetup}{\RaggedRight} % für linksbündig im LV


%punctfont=true

\setlength{\bibitemsep}{0cm}     % Abstand zwischen den Literaturangaben
\setlength{\bibhang}{1cm}        % Einzug nach jeweils erster Zeile

\DeclareRobustCommand{\a}{\autocite}
\DefineBibliographyStrings{german}{ibidem = {ebd\adddot}} % für die ibid Option

\defbibheading{bibliography}[Literaturverzeichnis]{%
	\chapter*{#1} \markboth{#1}{#1}
}                                              % Titel des LV
%%\defbibheading{subbibliography}[\refname]{\section*{#1}\vspace{0.5em}}
%%\defbibheading{subsubbibliography}[\refname]{\subsection*{#1}}

%\renewcommand{\bibfont}{\normalfont\small} % Schriftgrösse des LV

\defbibheading{subbibliography}[\refname]{\section*{#1}}
\defbibheading{subsubbibliography}[\refname]{\subsection*{#1}}

% Damit die family Befehle auch in french funktionieren:
\DefineBibliographyExtras{french}{\restorecommand\mkbibnamefamily}

% erst Nachname, dann Vorname
%\DeclareNameAlias{default}{last-first}, oder?:
\DeclareNameAlias{sortname}{family-given} % im LV
\DeclareNameAlias{default}{family-given} % in den FN
% Trenner zwischen den Namen ein Semikolon
\renewcommand*{\multinamedelim}{\addsemicolon\space}
% Semikolon statt und beim letzten:
%\renewcommand*{\finalnamedelim}{\addsemicolon\space}
% Doppelpunkt nach dem letzten Namen
%\renewcommand*{\labelnamepunct}{\addcolon\space}

% Autorennamen durch Schrägstrich oder Komma bzw. und trennen:
%\renewcommand*{\multinamedelim}{\addslash}
%\renewcommand*{\finalnamedelim}{\addslash}
\renewcommand*{\multinamedelim}{\addcomma\addspace}
\renewcommand*\finalnamedelim{%
	\ifnumgreater{\value{liststop}}{2}{\finalandcomma}{}%
	\addspace\bibstring{and}\space}%

% für alle anderen multiplen Einträge
\renewcommand*{\multilistdelim}{\addslash}
\renewcommand*{\finallistdelim}{\addslash}

%\DeclareFieldFormat[book]{title}{#1,}
\renewcommand*{\mkbibnamefamily}[1]{\textsc{#1}} % Kapitälchen oder nicht
\renewcommand{\labelnamepunct}{\addcomma\addspace} % Komma nach Vornamen
\renewcommand{\subtitlepunct}{\adddot\addspace} % Punkt zwischen titel und untertitel


%\renewbibmacro*{title}{%
%  \ifthenelse{\iffieldundef{title}\AND\iffieldundef{subtitle}}
%    {}
%    {\printtext[title]{%
%       \printfield[titlecase]{title}}%
%		   \setunit{\subtitlepunct}%
%     \printfield{subtitle}%
%     \newunit}%
%  \printfield{titleaddon}} % Subtitel als eigenes Feld


\renewcommand*{\newunitpunct}{\addcomma\addspace} % Komma statt Punkt nach einzelnen Elementen der Literaturangaben

\DefineBibliographyStrings{german}{%
	andothers = {u\adddot\ a\adddot},
	editor = {Hg\adddot}, % wenn der Editor am Anfang steht
	editors = {Hg\adddot},
	byeditor = {hg\adddot~von }, % wenn der Editor hinten steht
}

\renewbibmacro*{publisher+location+date}{%
	\printlist{location}%
	\iflistundef{publisher}
	{\setunit*{\addcomma\space}}
	{\setunit*{\addcolon\space}}% kein Komma nach Publisher bzw. Ort, je nach:
	%%\printlist{publisher}% ohne oder mit Publisher
	\setunit*{\addspace}% vorher: \setunit*{\addcomma\space}
	\usebibmacro{date}%
	\newunit}


\DeclareFieldFormat{part}{%
	\iffieldundef{volume}
	{#1}
	{\ifnumerals{#1}
		{,#1} % Komma nach volume, wenn part eine Nummer
		{\addcomma\addspace#1}} % Komma und Space nach volume, wenn part ein Text
}
% Noch besser: Nur das volume-Feld besetzen, dafür:
\DeclareFieldFormat[book]{volume}{#1}
\DeclareFieldFormat[article]{volume}{#1}
\DeclareFieldFormat[bookinbook]{volume}{#1}
\DeclareFieldFormat[incollection]{volume}{#1}
\DeclareFieldFormat[inProceedings]{volume}{#1}
\DeclareFieldFormat[inreference]{volume}{#1}

\DeclareFieldFormat{series}{\iffieldundef{number}{\mkbibparens{#1}}{\bibleftparen % Klammern um series und/ohne number
		#1}}
\DeclareFieldFormat{number}{#1\bibrightparen}

\DeclareFieldFormat{issuesubtitle}{#1\upshape}


%\DeclareFieldFormat[book]{location}{#1}
%\DeclareFieldFormat[book]{page}{#1}
%\DeclareFieldFormat[article]{postnote}{S. #1} % vor Seitenzahlen
%\DeclareFieldFormat[book]{multipostnote}{#1}

\makeatletter
\renewbibmacro*{bbx:editor}[1]{%
	\ifthenelse{\ifuseeditor\AND\NOT\ifnameundef{editor}}
	{\ifthenelse{\iffieldequals{fullhash}{\bbx@lasthash}\AND
			\NOT\iffirstonpage\AND
			\(\NOT\boolean{bbx@inset}\OR
			\iffieldequalstr{entrysetcount}{1}\)}
		{\bibnamedash}
		{\printnames{editor}%
			\setunit{\addspace}% GEÄNDERT
			\usebibmacro{bbx:savehash}}%
		\printtext[parens]{\usebibmacro{#1}}% GEÄNDERT % Für Klammer um Hrsg.
		\clearname{editor}%
		\setunit{\addspace},}% plus Komma nach Klammer
	{\global\undef\bbx@lasthash
		\usebibmacro{labeltitle}%
		\setunit*{\addspace}}%
	%\usebibmacro{date+extrayear}
}
\makeatother


%\begin{comment} % Damit Translator etc. nicht in Kapitälchen sind. Problem: dann auch Editor nicht, und funzt bei Foreword nicht
%\renewcommand*{\mkbibnamefamily}[1]{%
% \ifmknamesc{\textsc{#1}}{#1}}

%\renewcommand*{\mkbibnameprefix}[1]{%
% \ifboolexpr{ test {\ifmknamesc} and test {\ifuseprefix} }
%   {\textsc{#1}}
% {#1}}

%\def\ifmknamesc{%
%\ifboolexpr{ test {\ifcurrentname{labelname}}
%           or test {\ifcurrentname{author}}
%         or ( test {\ifnameundef{author}} and test {\ifcurrentname{editor}} ) }}
%\end{comment}


\DeclareBibliographyDriver{book}{%
	\usebibmacro{bibindex}%
	\usebibmacro{begentry}%
	\usebibmacro{author/editor+others/translator+others}%
	\setunit{\labelnamepunct}\newblock
	\usebibmacro{maintitle+title}%
	\newunit
	\printlist{language}%
	\newunit\newblock
	\usebibmacro{byauthor}%
	\newunit\newblock
	\usebibmacro{byeditor+others}%
	\newunit\newblock
	\printfield{edition}%
	\newunit
	\iffieldundef{maintitle}
	\newunit\newblock
	\setunit{\addspace} % Kein Komma vor Reihentitel
    \usebibmacro{series+number}%
	\newunit\newblock
	\usebibmacro{publisher+location+date}%
	\newunit
	\printfield{note}%
	\newunit\newblock
	\usebibmacro{chapter+pages}%
	\newunit
	\printfield{pagetotal}%
	\newunit\newblock
	\iftoggle{bbx:isbn}
	{\printfield{isbn}}
	{}%
	\newunit\newblock
	\usebibmacro{doi+eprint+url}%
	\newunit\newblock
	\usebibmacro{addendum+pubstate}%
	\setunit{\bibpagerefpunct}\newblock
	\usebibmacro{pageref}%
	\newunit\newblock
	\iftoggle{bbx:related}
	{\usebibmacro{related:init}%
		\usebibmacro{related}}
	{}%
	\usebibmacro{finentry}}


\DeclareBibliographyDriver{incollection}{%
	\usebibmacro{bibindex}%
	\usebibmacro{begentry}%
	\usebibmacro{author/translator+others}%
	\setunit{\labelnamepunct}\newblock
	\usebibmacro{title}%
	\newunit
	\printlist{language}%
	\newunit\newblock
	\usebibmacro{byauthor}%
	\newunit\newblock
	\usebibmacro{in:}%
	\usebibmacro{maintitle+booktitle}%
	\newunit
	\usebibmacro{byeditor+others}
	\newunit\newblock
	\iffieldundef{maintitle}
	{\printfield{volume}%
		\printfield{part}}
	{}%
	\setunit{\addspace}
	\usebibmacro{series+number}%
	\newunit\newblock
	\printfield{note}%
	\newunit\newblock
	\usebibmacro{publisher+location+date}%
	\newunit\newblock
	\usebibmacro{doi+eprint+url}%
	\newunit\newblock
	\usebibmacro{addendum+pubstate}%
	\newblock
	\usebibmacro{chapter+pages}%
	\setunit{\bibpagerefpunct}\newblock
	\usebibmacro{pageref}%
	\setunit{\bibpagespunct}%
	\usebibmacro{finentry}}


\DeclareBibliographyDriver{inproceedings}{%
	\usebibmacro{bibindex}%
	\usebibmacro{begentry}%
	\usebibmacro{author/translator+others}%
	\setunit{\labelnamepunct}\newblock
	\usebibmacro{title}%
	\newunit
	\printlist{language}%
	\newunit\newblock
	\usebibmacro{byauthor}%
	\newunit\newblock
	\usebibmacro{in:}%
	\usebibmacro{maintitle+booktitle}%
	\newunit\newblock
	\usebibmacro{event+venue+date}%
	\newunit\newblock
	\usebibmacro{byeditor+others}%
	\newunit\newblock
	\iffieldundef{maintitle}
	{\printfield{volume}%
		\printfield{part}}
	{}%
	\newunit
	\printfield{volumes}%
	\setunit{\addspace}
	\usebibmacro{series+number}%
	\newunit\newblock
	\printfield{note}%
	\newunit\newblock
	\printlist{organization}%
	\newunit
	\usebibmacro{publisher+location+date}%
	\newunit\newblock
	\usebibmacro{chapter+pages}%
	\newunit\newblock
	\iftoggle{bbx:isbn}
	{\printfield{isbn}}
	{}%
	\newunit\newblock
	\usebibmacro{doi+eprint+url}%
	\newunit\newblock
	\usebibmacro{addendum+pubstate}%
	\setunit{\bibpagerefpunct}\newblock
	\usebibmacro{pageref}%
	\usebibmacro{finentry}}

% Formatierung der Titel (kursiv oder nicht):
\DeclareFieldFormat[book]{title}{\textit{#1}}
\DeclareFieldFormat[book]{citetitle}{\textup{#1}}
\DeclareFieldFormat[article]{title}{\textup{#1}}
\DeclareFieldFormat[article]{citetitle}{\textup{#1}}
\DeclareFieldFormat[article]{issuetitle}{\textup{#1}}
\DeclareFieldFormat[incollection]{title}{\textup{#1}}
\DeclareFieldFormat[incollection]{citetitle}{\textup{#1}}
\DeclareFieldFormat[inproceedings]{title}{\textup{#1}}
\DeclareFieldFormat[inproceedings]{citetitle}{\textup{#1}}
\DeclareFieldFormat[bookinbook]{title}{\textup{#1}}
\DeclareFieldFormat[bookinbook]{citetitle}{\textup{#1}}
\DeclareFieldFormat{booktitle}{#1\isdot}
\DeclareFieldFormat{maintitle}{#1\isdot}
\DeclareFieldFormat{journaltitle}{#1\isdot}
\DeclareFieldFormat[online]{title}{\textup{#1}}
\DeclareFieldFormat[online]{citetitle}{\textup{#1}}
%für Anführungszeichen: \DeclareFieldFormat[article]{title}{\mkbibquote{#1}}

\begin{comment}
\DeclareSortingScheme{Reihenfolge}{ %Reihenfolge der Einträge im LV (als Paket-Option: sorting=Reihenfolge)
	\sort{\name{author}\name{editor}\name{sortname}}
	\sort{\field{maintitle}}
	\sort{\field[padside=left,padwidth=2,padchar=0]{volume}} % Optionen damit eine 0 vor einstelligen Zahlen für die alphabetische Reihenfolge
	\sort{\field[padside=left,padwidth=2,padchar=0]{part}}
	\sort{\field{booktitle}}
	\sort{\field{title}}
	\sort{\field{subtitle}}
	\sort{\field{year}}
}
\end{comment}


\urlstyle{same} % gleiche Schriftart für URL
\DeclareFieldFormat{url}{\url{#1}} % Ohne URL:
\DeclareFieldFormat{urldate}{\mkbibparens{#1}} % Ohne besucht am...

\DeclareSourcemap{% Ohne urldate
	\maps[datatype=bibtex]{
		\map{
			\step[fieldset=urldate, null]
		}
	}
}


%\DeclareFieldFormat{edition}{\textsuperscript{#1}} % Auflage als hochgestellt


\begin{comment}
% Damit das Originaljahr in eckigen Klammern im Zitat erscheint
\DeclareFieldFormat{origdate}{\mkbibbrackets{#1}}
\renewbibmacro*{cite}{%
\iffieldundef{shorthand}
{\ifthenelse{\ifciteibid\AND\NOT\iffirstonpage} % Ergänzung für ibid.
{\usebibmacro{cite:ibid}} % Ergänzung für ibid.
{\ifthenelse{\ifnameundef{labelname}\OR\iffieldundef{labelyear}}
{\usebibmacro{cite:label}%
\setunit{\printdelim{nonameyeardelim}}}
{\printnames{labelname}%
\setunit{\printdelim{nameyeardelim}}}%
\printorigdate
\setunit*{\addspace}
\usebibmacro{cite:labeldate+extradate}}} % 1 Klammer Ergänzung für ibid.
{\usebibmacro{cite:shorthand}}}
\end{comment}

% Für Originaljahr im LV
\renewbibmacro*{date}{%
	\iffieldundef{origyear}{%
	}{%
		\setunit*{\addspace}%
		\printtext[brackets]{\printorigdate}%
	}%
}






%Index
%%%%%%%%%%%%%%%%%%%%%%%%%%%%%%%%%%%%%%%%%%%%%%%%%%%%%

\usepackage{imakeidx}
\makeindex[name=aut,title=Personen] % Personenverzeichnis
\makeindex[name=res,title=Sachen] % Sachverzeichnis
\makeindex[name=bibel,title=Bibelstellen] % Stellenverzeichnis Bibel
% \index[aut]{Personenname} direkt in den Text hinter den Personenname, z.B.: “Die Ideenlehre\index[res]{Ideenlehre} Platons\index[aut]{Platon} wurde von Aristoteles\index[aut]{Aristoteles}, einem seiner Schüler, überwunden.”
% “\printindex[aut]” bzw. “\printindex[res]” für das Erzeugen der Verzeichnisse im Hauptdokument

\usepackage{balance} % um die beiden Spalten auf der letzten Seite des Index bündig zu machen. Dazu muss \balance irgendwo bei einem Eintrag in der linken Spalte stehen (z.B.: \index[aut]{Platon\balance})
